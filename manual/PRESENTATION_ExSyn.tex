
\section{Yosys by example -- Synthesis}

\begin{frame}
\sectionpage
\end{frame}

%%%%%%%%%%%%%%%%%%%%%%%%%%%%%%%%%%%%%%%%%%%%%%%%%%%%%%%%%%%%%%%%%%%%%%%%%%%%%

\subsection{Typical Phases of a Synthesis Flow}

\begin{frame}{\subsecname}
\begin{itemize}
\item Reading and elaborating the design
\item High-level synthesis and optimization
\begin{itemize}
\item Converting {\tt always}-blocks to logic and registers
\item Perform coarse-grain optimizations (resource sharing, const folding, ...)
\item Handling of memories and other coarse-grain blocks
\item Extracting and optimizing finite state machines
\end{itemize}
\item Convert remaining logic to bit-level logic functions
\item Perform optimizations on bit-level logic functions
\item Map bit-level logic and register to gates from cell library
\item Write results to output file 
\end{itemize}
\end{frame}

%%%%%%%%%%%%%%%%%%%%%%%%%%%%%%%%%%%%%%%%%%%%%%%%%%%%%%%%%%%%%%%%%%%%%%%%%%%%%

\subsection{Reading the design}

\begin{frame}[fragile]{\subsecname}
\begin{lstlisting}[xleftmargin=0.5cm, basicstyle=\ttfamily\fontsize{8pt}{10pt}\selectfont]
read_verilog file1.v
read_verilog -I include_dir -D enable_foo -D WIDTH=12 file2.v
read_verilog -lib cell_library.v

verilog_defaults -add -I include_dir
read_verilog file3.v
read_verilog file4.v
verilog_defaults -clear

verilog_defaults -push
verilog_defaults -add -I include_dir
read_verilog file5.v
read_verilog file6.v
verilog_defaults -pop
\end{lstlisting}
\end{frame}

%%%%%%%%%%%%%%%%%%%%%%%%%%%%%%%%%%%%%%%%%%%%%%%%%%%%%%%%%%%%%%%%%%%%%%%%%%%%%

\subsection{Design elaboration}

\begin{frame}[fragile]{\subsecname}
During design elaboration Yosys figures out how the modules are hierarchically
connected. It also re-runs the AST parts of the Verilog frontend to create
all needed variations of parametric modules.

\bigskip
\begin{lstlisting}[xleftmargin=0.5cm, basicstyle=\ttfamily\fontsize{8pt}{10pt}\selectfont]
# simplest form. at least this version should be used after reading all input files
#
hierarchy

# recommended form. fail if parts of the design hierarchy are missing. remove
# everything that is unreachable by the top module. mark the top module.
#
hierarchy -check -top top_module
\end{lstlisting}
\end{frame}

%%%%%%%%%%%%%%%%%%%%%%%%%%%%%%%%%%%%%%%%%%%%%%%%%%%%%%%%%%%%%%%%%%%%%%%%%%%%%

\subsection{The ``proc'' commands}

\begin{frame}[fragile]{\subsecname}
The Verilog frontend converts {\tt always}-blocks to RTL netlists for the
expressions and ``processes'' for the control- and memory elements.

\medskip
The {\tt proc} command transforms this ``processes'' to netlists of RTL
multiplexer and register cells.

\medskip
The {\tt proc} command is actually a macro-command that calls the following
other commands:

\begin{lstlisting}[xleftmargin=0.5cm, basicstyle=\ttfamily\fontsize{8pt}{10pt}\selectfont]
proc_clean      # remove empty branches and processes
proc_rmdead     # remove unreachable branches
proc_init       # special handling of "initial" blocks
proc_arst       # identify modeling of async resets
proc_mux        # convert decision trees to multiplexer networks
proc_dff        # extract registers from processes
proc_clean      # if all went fine, this should remove all the processes
\end{lstlisting}

\medskip
Many commands can not operate on modules with ``processes'' in them. Usually
a call to {\tt proc} is the first command in the actual synthesis procedure
after design elaboration.
\end{frame}

\begin{frame}[fragile]{\subsecname{} -- Example 1/3}
\begin{columns}
\column[t]{5cm}
\lstinputlisting[basicstyle=\ttfamily\fontsize{8pt}{10pt}\selectfont, language=verilog]{PRESENTATION_ExSyn/proc_01.v}
\column[t]{5cm}
\lstinputlisting[basicstyle=\ttfamily\fontsize{8pt}{10pt}\selectfont, frame=single]{PRESENTATION_ExSyn/proc_01.ys}
\end{columns}
\hfil\includegraphics[width=8cm,trim=0 0cm 0 0cm]{PRESENTATION_ExSyn/proc_01.pdf}
\end{frame}

\begin{frame}[t, fragile]{\subsecname{} -- Example 2/3}
\vbox to 0cm{\includegraphics[width=\linewidth,trim=0cm 0cm 0cm -2.5cm]{PRESENTATION_ExSyn/proc_02.pdf}\vss}
\vskip-1cm
\begin{columns}
\column[t]{5cm}
\lstinputlisting[basicstyle=\ttfamily\fontsize{8pt}{10pt}\selectfont, language=verilog]{PRESENTATION_ExSyn/proc_02.v}
\column[t]{5cm}
\lstinputlisting[basicstyle=\ttfamily\fontsize{8pt}{10pt}\selectfont, frame=single]{PRESENTATION_ExSyn/proc_02.ys}
\end{columns}
\end{frame}

\begin{frame}[t, fragile]{\subsecname{} -- Example 3/3}
\vbox to 0cm{\includegraphics[width=\linewidth,trim=0cm 0cm 0cm -1.5cm]{PRESENTATION_ExSyn/proc_03.pdf}\vss}
\vskip-1cm
\begin{columns}
\column[t]{5cm}
\lstinputlisting[basicstyle=\ttfamily\fontsize{8pt}{10pt}\selectfont, frame=single]{PRESENTATION_ExSyn/proc_03.ys}
\column[t]{5cm}
\lstinputlisting[basicstyle=\ttfamily\fontsize{8pt}{10pt}\selectfont, language=verilog]{PRESENTATION_ExSyn/proc_03.v}
\end{columns}
\end{frame}

%%%%%%%%%%%%%%%%%%%%%%%%%%%%%%%%%%%%%%%%%%%%%%%%%%%%%%%%%%%%%%%%%%%%%%%%%%%%%

\subsection{The ``opt'' commands}

\begin{frame}[fragile]{\subsecname}
The {\tt opt} command implements a series of simple optimizations. It also
is a macro command that calls other commands:

\begin{lstlisting}[xleftmargin=0.5cm, basicstyle=\ttfamily\fontsize{8pt}{10pt}\selectfont]
opt_const               # const folding
opt_share -nomux        # merging identical cells

do
    opt_muxtree         # remove never-active branches from multiplexer tree
    opt_reduce          # consolidate trees of boolean ops to reduce functions
    opt_share           # merging identical cells
    opt_rmdff           # remove/simplify registers with constant inputs
    opt_clean           # remove unused objects (cells, wires) from design
    opt_const           # const folding
while [changed design]
\end{lstlisting}

The command {\tt clean} can be used as alias for {\tt opt\_clean}. And {\tt ;;}
can be used as shortcut for {\tt clean}. For example:

\begin{lstlisting}[xleftmargin=0.5cm, basicstyle=\ttfamily\fontsize{8pt}{10pt}\selectfont]
proc; opt; memory; opt_const;; fsm;;
\end{lstlisting}
\end{frame}

\begin{frame}[t, fragile]{\subsecname{} -- Example 1/4}
\vbox to 0cm{\includegraphics[width=\linewidth,trim=0cm 0cm 0cm -0.5cm]{PRESENTATION_ExSyn/opt_01.pdf}\vss}
\vskip-1cm
\begin{columns}
\column[t]{5cm}
\lstinputlisting[basicstyle=\ttfamily\fontsize{8pt}{10pt}\selectfont, frame=single]{PRESENTATION_ExSyn/opt_01.ys}
\column[t]{5cm}
\lstinputlisting[basicstyle=\ttfamily\fontsize{8pt}{10pt}\selectfont, language=verilog]{PRESENTATION_ExSyn/opt_01.v}
\end{columns}
\end{frame}

\begin{frame}[t, fragile]{\subsecname{} -- Example 2/4}
\vbox to 0cm{\includegraphics[width=\linewidth,trim=0cm 0cm 0cm 0cm]{PRESENTATION_ExSyn/opt_02.pdf}\vss}
\vskip-1cm
\begin{columns}
\column[t]{5cm}
\lstinputlisting[basicstyle=\ttfamily\fontsize{8pt}{10pt}\selectfont, frame=single]{PRESENTATION_ExSyn/opt_02.ys}
\column[t]{5cm}
\lstinputlisting[basicstyle=\ttfamily\fontsize{8pt}{10pt}\selectfont, language=verilog]{PRESENTATION_ExSyn/opt_02.v}
\end{columns}
\end{frame}

\begin{frame}[t, fragile]{\subsecname{} -- Example 3/4}
\vbox to 0cm{\includegraphics[width=\linewidth,trim=0cm 0cm 0cm -2cm]{PRESENTATION_ExSyn/opt_03.pdf}\vss}
\vskip-1cm
\begin{columns}
\column[t]{5cm}
\lstinputlisting[basicstyle=\ttfamily\fontsize{8pt}{10pt}\selectfont, frame=single]{PRESENTATION_ExSyn/opt_03.ys}
\column[t]{5cm}
\lstinputlisting[basicstyle=\ttfamily\fontsize{8pt}{10pt}\selectfont, language=verilog]{PRESENTATION_ExSyn/opt_03.v}
\end{columns}
\end{frame}

\begin{frame}[t, fragile]{\subsecname{} -- Example 4/4}
\vbox to 0cm{\hskip6cm\includegraphics[width=6cm,trim=0cm 0cm 0cm -3cm]{PRESENTATION_ExSyn/opt_04.pdf}\vss}
\vskip-1cm
\begin{columns}
\column[t]{5cm}
\lstinputlisting[basicstyle=\ttfamily\fontsize{8pt}{10pt}\selectfont, language=verilog]{PRESENTATION_ExSyn/opt_04.v}
\column[t]{5cm}
\lstinputlisting[basicstyle=\ttfamily\fontsize{8pt}{10pt}\selectfont, frame=single]{PRESENTATION_ExSyn/opt_04.ys}
\end{columns}
\end{frame}

%%%%%%%%%%%%%%%%%%%%%%%%%%%%%%%%%%%%%%%%%%%%%%%%%%%%%%%%%%%%%%%%%%%%%%%%%%%%%

\subsection{When to use ``opt'' or ``clean''}

\begin{frame}{\subsecname}
Usually it does not hurt to call {\tt opt} after each regular command in the
synthesis script. But it increases the synthesis time, so it is favourable
to only call {\tt opt} when an improvement can be archieved.

\bigskip
The designs in {\tt yosys-bigsim} are a good playground for experimenting with
the effects of calling {\tt opt} in various places of the flow.

\bigskip
It generally is a good idea us call {\tt opt} before inherently expensive
commands such as {\tt sat} or {\tt freduce}, as the possible gain is much
higher in this cases as the possible loss.

\bigskip
The {\tt clean} command on the other hand is very fast and many commands leave
a mess (dangling signal wires, etc). For example, most commands do not remove
any wires or cells. They just change the connections and depend on a later
call to clean to get rid of the now unused objects. So the occasional {\tt ;;}
is a good idea in every synthesis script.
\end{frame}

%%%%%%%%%%%%%%%%%%%%%%%%%%%%%%%%%%%%%%%%%%%%%%%%%%%%%%%%%%%%%%%%%%%%%%%%%%%%%

\subsection{The ``memory'' commands}

\begin{frame}[fragile]{\subsecname}
In the RTL netlist, memory reads and writes are individual cells. This makes
consolidating the number of ports for a memory easier. The {\tt memory}
transforms memories to an implementation. Per default that is logic for address
decoders and registers. It also is a macro command that calls other commands:

\begin{lstlisting}[xleftmargin=0.5cm, basicstyle=\ttfamily\fontsize{8pt}{10pt}\selectfont]
# this merges registers into the memory read- and write cells.
memory_dff

# this collects all read and write cells for a memory and transforms them
# into one multi-port memory cell.
memory_collect

# this takes the multi-port memory cells and transforms it to address decoder
# logic and registers. This step is skipped if "memory" is called with -nomap.
memory_map
\end{lstlisting}

\bigskip
Usually it is preferred to use architecture-specific RAM resources for memory.
For example:

\begin{lstlisting}[xleftmargin=0.5cm, basicstyle=\ttfamily\fontsize{8pt}{10pt}\selectfont]
memory -nomap; techmap -map my_memory_map.v; memory_map
\end{lstlisting}
\end{frame}

\begin{frame}[t, fragile]{\subsecname{} -- Example 1/2}
\vbox to 0cm{\includegraphics[width=\linewidth,trim=0cm 0cm 0cm -10cm]{PRESENTATION_ExSyn/memory_01.pdf}\vss}
\vskip-1cm
\begin{columns}
\column[t]{5cm}
\lstinputlisting[basicstyle=\ttfamily\fontsize{8pt}{10pt}\selectfont, frame=single]{PRESENTATION_ExSyn/memory_01.ys}
\column[t]{5cm}
\lstinputlisting[basicstyle=\ttfamily\fontsize{8pt}{10pt}\selectfont, language=verilog]{PRESENTATION_ExSyn/memory_01.v}
\end{columns}
\end{frame}

\begin{frame}[t, fragile]{\subsecname{} -- Example 2/2}
\vbox to 0cm{\hfill\includegraphics[width=7.5cm,trim=0cm 0cm 0cm -6cm]{PRESENTATION_ExSyn/memory_02.pdf}\vss}
\vskip-1cm
\begin{columns}
\column[t]{5cm}
\lstinputlisting[basicstyle=\ttfamily\fontsize{6pt}{8pt}\selectfont, language=verilog]{PRESENTATION_ExSyn/memory_02.v}
\column[t]{5cm}
\lstinputlisting[basicstyle=\ttfamily\fontsize{8pt}{10pt}\selectfont, frame=single]{PRESENTATION_ExSyn/memory_02.ys}
\end{columns}
\end{frame}

%%%%%%%%%%%%%%%%%%%%%%%%%%%%%%%%%%%%%%%%%%%%%%%%%%%%%%%%%%%%%%%%%%%%%%%%%%%%%

\subsection{The ``fsm'' commands}

\begin{frame}[fragile]{\subsecname{}}
The {\tt fsm} command identifies, extracts, optimizes (re-encodes), and
re-synthesizes finite state machines. It again is a macro that calls
a series of other commands:

\begin{lstlisting}[xleftmargin=0.5cm, basicstyle=\ttfamily\fontsize{8pt}{10pt}\selectfont]
fsm_detect          # unless got option -nodetect
fsm_extract

fsm_opt
opt_clean
fsm_opt

fsm_expand          # if got option -expand
opt_clean           # if got option -expand
fsm_opt             # if got option -expand

fsm_recode          # unless got option -norecode

fsm_info

fsm_export          # if got option -export
fsm_map             # unless got option -nomap
\end{lstlisting}
\end{frame}

\begin{frame}{\subsecname{} -- details}
Some details on the most importand commands from the {\tt fsm\_*} group:

\bigskip
The {\tt fsm\_detect} command identifies FSM state registers and marks them
with the {\tt (* fsm\_encoding = "auto" *)} attribute, if they do not have the
{\tt fsm\_encoding} set already. Mark registers with {\tt (* fsm\_encoding =
"none" *)} to disable FSM optimization for a register.

\bigskip
The {\tt fsm\_extract} command replaces the entire FSM (logic and state
registers) with a {\tt \$fsm} cell.

\bigskip
The commands {\tt fsm\_opt} and {\tt fsm\_recode} can be used to optimize the
FSM.

\bigskip
Finally the {\tt fsm\_map} command can be used to convert the (optimized) {\tt
\$fsm} cell back to logic and registers.
\end{frame}

%%%%%%%%%%%%%%%%%%%%%%%%%%%%%%%%%%%%%%%%%%%%%%%%%%%%%%%%%%%%%%%%%%%%%%%%%%%%%

\subsection{The ``techmap'' command}

\begin{frame}[t]{\subsecname}
\vbox to 0cm{\includegraphics[width=12cm,trim=-18cm 0cm 0cm -34cm]{PRESENTATION_ExSyn/techmap_01.pdf}\vss}
\vskip-0.8cm
The {\tt techmap} command replaces cells with implementations given as
verilog source. For example implementing a 32 bit adder using 16 bit adders:

\vbox to 0cm{
\vskip-0.3cm
\lstinputlisting[basicstyle=\ttfamily\fontsize{6pt}{7pt}\selectfont, language=verilog]{PRESENTATION_ExSyn/techmap_01_map.v}
}\vbox to 0cm{
\vskip-0.5cm
\lstinputlisting[xleftmargin=5cm, basicstyle=\ttfamily\fontsize{8pt}{10pt}\selectfont, frame=single, language=verilog]{PRESENTATION_ExSyn/techmap_01.v}
\lstinputlisting[xleftmargin=5cm, basicstyle=\ttfamily\fontsize{8pt}{10pt}\selectfont, frame=single]{PRESENTATION_ExSyn/techmap_01.ys}
}
\end{frame}

\begin{frame}[t]{\subsecname{} -- stdcell mapping}
When {\tt techmap} is used without a map file, it uses a built-in map file
to map all RTL cell types to a generic library of built-in logic gates and registers.

\bigskip
\begin{block}{The build-in logic gate types are:}
{\tt \$\_INV\_ \$\_AND\_ \$\_OR\_ \$\_XOR\_ \$\_MUX\_}
\end{block}

\bigskip
\begin{block}{The register types are:}
{\tt \$\_SR\_NN\_ \$\_SR\_NP\_ \$\_SR\_PN\_ \$\_SR\_PP\_ \\
\$\_DFF\_N\_ \$\_DFF\_P\_ \\
\$\_DFF\_NN0\_ \$\_DFF\_NN1\_ \$\_DFF\_NP0\_ \$\_DFF\_NP1\_ \\
\$\_DFF\_PN0\_ \$\_DFF\_PN1\_ \$\_DFF\_PP0\_ \$\_DFF\_PP1\_ \\
\$\_DFFSR\_NNN\_ \$\_DFFSR\_NNP\_ \$\_DFFSR\_NPN\_ \$\_DFFSR\_NPP\_ \\
\$\_DFFSR\_PNN\_ \$\_DFFSR\_PNP\_ \$\_DFFSR\_PPN\_ \$\_DFFSR\_PPP\_ \\
\$\_DLATCH\_N\_ \$\_DLATCH\_P\_}
\end{block}
\end{frame}

%%%%%%%%%%%%%%%%%%%%%%%%%%%%%%%%%%%%%%%%%%%%%%%%%%%%%%%%%%%%%%%%%%%%%%%%%%%%%

\subsection{The ``abc'' command}

\begin{frame}{\subsecname}
The {\tt abc} command provides an interface to ABC\footnote[frame]{\url{http://www.eecs.berkeley.edu/~alanmi/abc/}},
an open source tool for low-level logic synthesis.

\medskip
The {\tt abc} command processes a netlist of internal gate types and can perform:
\begin{itemize}
\item logic minimization (optimization)
\item mapping of logic to standard cell library (liberty format)
\item mapping of logic to k-LUTs (for FPGA synthesis)
\end{itemize}

\medskip
Optionally {\tt abc} can process registers from one clock domain and perform
sequential optimization (such as register balancing).

\medskip
ABC is also controlled using scripts. An ABC script can be specified to use
more advanced ABC features. It is also possible to write the design with
{\tt write\_blif} and load the output file into ABC outside of Yosys.
\end{frame}

\begin{frame}[fragile]{\subsecname{} -- Example}
\begin{columns}
\column[t]{5cm}
\lstinputlisting[basicstyle=\ttfamily\fontsize{8pt}{10pt}\selectfont, language=verilog]{PRESENTATION_ExSyn/abc_01.v}
\column[t]{5cm}
\lstinputlisting[basicstyle=\ttfamily\fontsize{8pt}{10pt}\selectfont, frame=single]{PRESENTATION_ExSyn/abc_01.ys}
\end{columns}
\includegraphics[width=\linewidth,trim=0 0cm 0 0cm]{PRESENTATION_ExSyn/abc_01.pdf}
\end{frame}

%%%%%%%%%%%%%%%%%%%%%%%%%%%%%%%%%%%%%%%%%%%%%%%%%%%%%%%%%%%%%%%%%%%%%%%%%%%%%

\subsection{Other special-purpose mapping commands}

\begin{frame}{\subsecname}
\begin{block}{\tt dfflibmap}
This command maps the internal register cell types to the register types
described in a liberty file.
\end{block}

\bigskip
\begin{block}{\tt hilomap}
Some architectures require special driver cells for driving a constant hi or lo
value. This command replaces simple constants with instances of such driver cells.
\end{block}

\bigskip
\begin{block}{\tt iopadmap}
Top-level input/outputs must usually be implemented using special I/O-pad cells.
This command inserts this cells to the design.
\end{block}
\end{frame}

%%%%%%%%%%%%%%%%%%%%%%%%%%%%%%%%%%%%%%%%%%%%%%%%%%%%%%%%%%%%%%%%%%%%%%%%%%%%%

\subsection{Example Synthesis Script}

\begin{frame}[fragile]{\subsecname}
\begin{columns}
\column[t]{4cm}
\begin{lstlisting}[basicstyle=\ttfamily\fontsize{6pt}{7pt}\selectfont]
# read and elaborate design
read_verilog cpu_top.v cpu_ctrl.v cpu_regs.v
read_verilog -D WITH_MULT cpu_alu.v
hierarchy -check -top cpu_top

# high-level synthesis
proc; opt; memory -nomap;; fsm; opt

# substitute block rams
techmap -map map_rams.v

# map remaining memories
memory_map

# low-level synthesis
techmap; opt; flatten;; abc -lut6
techmap -map map_xl_cells.v

# add clock buffers
select -set xl_clocks t:FDRE %x:+FDRE[C] t:FDRE %d
iopadmap -inpad BUFGP O:I @xl_clocks

# add io buffers
select -set xl_nonclocks w:* t:BUFGP %x:+BUFGP[I] %d
iopadmap -outpad OBUF I:O -inpad IBUF O:I @xl_nonclocks

# write synthesis results
write_edif synth.edif
\end{lstlisting}
\column[t]{6cm}
\vskip1cm
\begin{block}{Teaser / Outlook}
\small\parbox{6cm}{
The weird {\tt select} expressions at the end of this script are discussed in
the next part (Section 3, ``Advanced Synthesis'') of this presentation.}
\end{block}
\end{columns}
\end{frame}

%%%%%%%%%%%%%%%%%%%%%%%%%%%%%%%%%%%%%%%%%%%%%%%%%%%%%%%%%%%%%%%%%%%%%%%%%%%%%

