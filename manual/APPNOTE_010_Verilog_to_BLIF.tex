
\appnote{010}{Converting Verilog to BLIF}{Clifford Wolf}

\begin{appnote_abstract}
Verilog-2005 is a powerful Hardware Description Language (HDL) that can be used
to easily create complex designs from small HDL code. It is the prefered
method of design entry for many designers\footnote{The other half prefers VHDL,
a very different but -- of course -- equaly powerful language.}.

The Berkeley Logic Interchange Format (BLIF) is a simple file format for
exchanging sequential logic between programs. It is easy to generate and
easy to parse and is therefore the prefered method of design entry for
many authors of logic synthesis tools.

Yosys\footnote{\url{http://www.clifford.at/yosys/}} is a feature-rich Open-Source Verilog synthesis tool that can be used to
bridge the gap between the two file formats. It implements most of Verilog-2005
and thus can be used to import modern behavioral Verilog designs into BLIF-based
design flows without dependencies on proprietary synthesis tools.
\end{appnote_abstract}

\section{Installation}

Yosys written in C++ (using features from C++11) and is tested on modern Linux.
It should compile fine on most UNIX systems with a C++11 compiler. The README
file contains useful information on building Yosys and its prerequisites.

Yosys is a large and feature-rich program with a couple of dependencies. It is,
however, possible to deactivate some of the dependencies in the Makefile,
resulting in features in Yosys becoming unavailable. When problems with building
Yosys are encountered, a user who is only interested in the features of Yosys
that are presented in this Application Note may deactivate {\tt TCL}, {\tt Qt}
and {\tt MiniSAT} support and not build {\tt yosys-abc}.

\bigskip

This Application Note is based on GIT Rev. {\color{red} FIXME} from
{\color{red} DATE} of Yosys. The Verilog sources used for the examples
is taken from the {\it yosys-bigsim test
bench}\footnote{\url{https://github.com/cliffordwolf/yosys-bigsim}}, GIT
Rev. {\color{red} FIXME}.

\section{Getting Started}

We start with the {\tt softusb\_navre} core from {\it yosys-bigsim}. The navre
processor\footnote{\url{http://opencores.org/project,navre}} is an Open Source
AVR clone. It is a single module ({\tt softusb\_navre}) in a single design file
({\tt softusb\_navre.v}). It also is using only features that map nicely to
the BLIF format, for example it only uses synchronous resets.

Converting {\tt softusb\_navre.v} to {\tt softusb\_navre.blif} could not be
easier:

\begin{lstlisting}[frame=trBL,xleftmargin=1.5em,numbers=left]
  yosys -o softusb_navre.blif \
                 -S softusb_navre.v
\end{lstlisting}

Behind the scenes Yosys is controlled by synthesis scripts that execute
commands that operate on Yosys' internal state. For example, the {\tt -o
softusb\_navre.blif} option just adds the command {\tt write\_blif
softusb\_navre.blif} to the end of the script. Likewise a file on the
command line -- {\tt softusb\_navre.v} in this case -- adds the command
{\tt read\_verilog softusb\_navre.v} to the beginning of the
synthesis script. In both cases the file type is detected from the
file extension.

Finally the option {\tt -S} instantiates a built-in default synthesis script.
Instead of using {\tt -S} one could also specify the synthesis commands
for the script on the command line using the {\tt -p} option, either using
individual options for each command or by passing one big command string
with semicolon-separated commands. But in most cases it is more convenient
to use an actual script file.

\section{Using a Synthesis Script}

With a script file we have better control over Yosys. The following script
file replicates what the command from the last section did:

\begin{lstlisting}[frame=trBL,xleftmargin=1.5em,numbers=left,caption={\tt softusb\_navre.ys}]
read_verilog softusb_navre.v
hierarchy
proc; opt; memory; opt; techmap; opt
write_blif softusb_navre.blif
\end{lstlisting}

The first and last line obviously read the Verilog file and write the BLIF
file. 

\medskip

The 2nd line checks the design hierarchy and instantiates parametrized 
versions of the modules in the design, if necessary. In the case of this
simple design this is a no-op. However, as a general rule a synthesis script
should always contain this command as first command after reading the input
files.

\medskip

The 3rd line does most of the actual work:

\begin{itemize}
\item The command {\tt opt} is the Yosys' built-in optimizer. It can perform
some simple optimizations such as const-folding and removing unconnected parts
of the design. It is common practice to call opt after each major step in the
synthesis. In cases where too much optimization is not appreciated (for example
when analyzing a design), it is recommended to call {\tt clean} instead of {\tt
opt}.
\item The command {\tt proc} converts {\it processes} (Yosys' internal
representation of Verilog {\tt always}- and {\tt initial}-blocks) to circuits
of multiplexers and storage elements (various types of flip-flops).
\item The command {\tt memory} converts Yosys' internal representation of 
arrays and array accesses to multi-port block memories, and then maps this
block memories to address decoders and flip-flops, unless the option {\tt -nomap}
is used, in which case the multi-port block memories stay in the design
and can then be mapped to architecture-specific memory primitives using
other commands.
\item The command {\tt techmap} turns a high-level circuit with coarse grain
cells such as wide adders and multipliers to a fine-grain circuit of simple
logic primitives and single-bit storage elements. The command does that by
substituting the complex cells by circuits of simpler cells. It is possible
to provide a custom set of rules for this process in the form of a Verilog
source file, as we will see in the next section.
\end{itemize}

Now Yosys can be run with the file of the synthesis script as argument:

\begin{lstlisting}[frame=trBL,xleftmargin=1.5em,numbers=left]
  yosys softusb_navre.ys
\end{lstlisting}

\medskip

Now that we are using a synthesis script we can easily modify how Yosys
synthesizes the design. The first thing we should customize is the
call to the {\tt history} command:

Whenever it is known that there are no implicit blackboxes in the design, i.e.
modules that are referred to but are not defined, the {\tt hierarchy} command
should be called with the {\tt -check} option. The 2nd thing we can improve
regarding the {\tt hierarchy} command is that we can tell it the name of the
top level module of the design hierarchy. It will then automatically remove
all modules that are not referenced from this top level module.

\medskip

For many designs it is also desired to optimize the encodings for the finite
state machines (FSM) in the design. The {\tt fsm command} finds FSMs, extracts
them, performs some basic optimizations and then generate a circuit from
the extracted and optimized description. It would also be possible to tell
the FSM command to leave the FSMs in their extracted form, so they can be
processed using custom commands. But in this case we don't need that.

\medskip

So now we have the final synthesis script for generating a BLIF file
for the navre CPU:

\begin{lstlisting}[frame=trBL,xleftmargin=1.5em,numbers=left,caption={\tt softusb\_navre.ys} (improved)]
read_verilog softusb_navre.v
hierarchy -check -top softusb_navre
proc; opt; memory; opt;
   fsm; opt; techmap; opt
write_blif softusb_navre.blif
\end{lstlisting}

\section{Advanced Example: The Amber23 ARMv2a CPU}

Our 2nd example is the Amber23\footnote{\url{http://opencores.org/project,amber}}
ARMv2a CPU. Once again we base our example on the Verilog code that is included
in {\it yosys-bigsim}.

\begin{lstlisting}[frame=trBL,xleftmargin=1.5em,numbers=left,caption={\tt amber23.ys}]
read_verilog a23_alu.v
read_verilog a23_barrel_shift_fpga.v
read_verilog a23_barrel_shift.v
read_verilog a23_cache.v
read_verilog a23_coprocessor.v
read_verilog a23_core.v
read_verilog a23_decode.v
read_verilog a23_execute.v
read_verilog a23_fetch.v
read_verilog a23_multiply.v
read_verilog a23_ram_register_bank.v
read_verilog a23_register_bank.v
read_verilog a23_wishbone.v
read_verilog generic_sram_byte_en.v
read_verilog generic_sram_line_en.v
hierarchy -check -top a23_core
add -global_input globrst 1
proc -global_arst globrst
opt; memory; opt; fsm; opt
techmap -map adff2dff.v
techmap
write_blif amber23.blif
\end{lstlisting}


