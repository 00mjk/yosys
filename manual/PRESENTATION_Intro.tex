
\section{Introduction}

\begin{frame}
\sectionpage
\end{frame}

%%%%%%%%%%%%%%%%%%%%%%%%%%%%%%%%%%%%%%%%%%%%%%%%%%%%%%%%%%%%%%%%%%%%%%%%%%%%%

\subsection{Representations of (digital) Circuits}

\begin{frame}[t]{\subsecname}
\begin{itemize}
	\item Graphical
		\begin{itemize}
			\item \alert<1>{Schematic Diagram}
			\item \alert<2>{Physical Layout}
		\end{itemize}
	\bigskip
	\item Non-graphical
		\begin{itemize}
			\item \alert<3>{Netlists}
			\item \alert<4>{Hardware Description Languages (HDLs)}
		\end{itemize}
\end{itemize}
\bigskip
\begin{block}{Definition:
\only<1>{Schematic Diagram}%
\only<2>{Physical Layout}%
\only<3>{Netlists}%
\only<4>{Hardware Description Languages (HDLs)}}
	\only<1>{TBD}
	\only<2>{TBD}
	\only<3>{TBD}
	\only<4>{TBD}
\end{block}
\end{frame}

%%%%%%%%%%%%%%%%%%%%%%%%%%%%%%%%%%%%%%%%%%%%%%%%%%%%%%%%%%%%%%%%%%%%%%%%%%%%%

\subsection{Levels of Abstraction for Digital Circuits}

\begin{frame}[t]{\subsecname}
\begin{itemize}
	\item \alert<1>{System Level}
	\item \alert<2>{High Level}
	\item \alert<3>{Behavioral Level}
	\item \alert<4>{Register-Transfer Level (RTL)}
	\item \alert<5>{Logical Gate Level}
	\item \alert<6>{Physical Gate Level}
	\item \alert<7>{Switch Level}
\end{itemize}
\bigskip
\begin{block}{Definition:
\only<1>{System Level}%
\only<2>{High Level}%
\only<3>{Behavioral Level}%
\only<4>{Register-Transfer Level (RTL)}%
\only<5>{Logical Gate Level}%
\only<6>{Physical Gate Level}%
\only<7>{Switch Level}}
\only<1>{
	Overall view of the circuit: E.g. block-diagrams or instruction-set architecture descriptions
}%
\only<2>{
	Functional implementation of circuit in high-level programming language (C, C++, SystemC, Matlab, Python, etc.).
}%
\only<3>{
	Cycle-accurate description of circuit in hardware description language (Verilog, VHDL, etc.).
}%
\only<4>{
	List of registers (flip-flops) and logic functions that calculate the next state from the previous one. Usually
	a netlist utilizing high-level cells such as adders, multiplieres, multiplexer, etc.
}%
\only<5>{
	Netlist of single-bit registers and basic logic gates (such as AND, OR,
	NOT, etc.). Popular form: And-Inverter-Graphs (AIGs) with pairs of primary
	inputs and outputs for each register bit.
}%
\only<6>{
	Netlist of cells that actually are available on the target architecture
	(such as CMOS gates in an ASCI or LUTs in an FPGA). Optimized for
	area and/or and/or speed (static timing or number of logic levels).
}%
\only<7>{
	Netlist of individual transistors.
}%
\end{block}
\end{frame}

%%%%%%%%%%%%%%%%%%%%%%%%%%%%%%%%%%%%%%%%%%%%%%%%%%%%%%%%%%%%%%%%%%%%%%%%%%%%%

\subsection{Digital Circuit Synthesis}

\begin{frame}{\subsecname}
\end{frame}

