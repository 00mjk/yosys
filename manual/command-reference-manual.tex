% Generated using the yosys 'help -write-tex-command-reference-manual' command.

\section{abc -- use ABC for technology mapping}
\label{cmd:abc}
\begin{lstlisting}[numbers=left,frame=single]
    abc [options] [selection]

This pass uses the ABC tool [1] for technology mapping of yosys's internal gate
library to a target architecture.

    -exe <command>
        use the specified command name instead of "yosys-abc" to execute ABC.
        This can e.g. be used to call a specific version of ABC or a wrapper.

    -script <file>
        use the specified ABC script file instead of the default script.

    -liberty <file>
        generate netlists for the specified cell library (using the liberty
        file format). Without this option, ABC is used to optimize the netlist
        but keeps using yosys's internal gate library. This option is ignored if
        the -script option is also used.

    -constr <file>
        pass this file with timing constraints to ABC

    -lut <width>
        generate netlist using luts of (max) the specified width.

    -nocleanup
        when this option is used, the temporary files created by this pass
        are not removed. this is useful for debugging.

This pass does not operate on modules with unprocessed processes in it.
(I.e. the 'proc' pass should be used first to convert processes to netlists.)

[1] http://www.eecs.berkeley.edu/~alanmi/abc/
\end{lstlisting}

\section{add -- add objects to the design}
\label{cmd:add}
\begin{lstlisting}[numbers=left,frame=single]
    add <command> [selection]

This command adds objects to the design. It operates on all fully selected
modules. So e.g. 'add -wire foo' will add a wire foo to all selected modules.


    add {-wire|-input|-inout|-output} <name> <width> [selection]

Add a wire (input, inout, output port) with the given name and width. The
command will fail if the object exists already and has different properties
than the object to be created.


    add -global_input <name> <width> [selection]

Like 'add -input', but also connect the signal between instances of the
selected modules.
\end{lstlisting}

\section{cd -- a shortcut for 'select -module <name>'}
\label{cmd:cd}
\begin{lstlisting}[numbers=left,frame=single]
    cd <modname>

This is just a shortcut for 'select -module <modname>'.


    cd <cellname>

When no module with the specified name is found, but there is a cell
with the specified name in the current module, then this is equivialent
to 'cd <celltype>'.

    cd ..

This is just a shortcut for 'select -clear'.
\end{lstlisting}

\section{clean -- remove unused cells and wires}
\label{cmd:clean}
\begin{lstlisting}[numbers=left,frame=single]
    clean [options] [selection]

This is identical to 'opt_clean', but less verbose.

When commands are seperated using the ';;' token, this command will be executed
between the commands.

When commands are seperated using the ';;;' token, this command will be executed
in -purge mode between the commands.
\end{lstlisting}

\section{design -- save, restore and reset current design}
\label{cmd:design}
\begin{lstlisting}[numbers=left,frame=single]
    design -reset

Clear the current design.


    design -save <name>

Save the current design under the given name.


    design -load <name>

Reset the current design and load the design previously saved under the given
name.
\end{lstlisting}

\section{dfflibmap -- technology mapping of flip-flops}
\label{cmd:dfflibmap}
\begin{lstlisting}[numbers=left,frame=single]
    dfflibmap -liberty <file> [selection]

Map internal flip-flop cells to the flip-flop cells in the technology
library specified in the given liberty file.

This pass may add inverters as needed. Therefore it is recommended to
first run this pass and then map the logic paths to the target technology.
\end{lstlisting}

\section{dump -- print parts of the design in ilang format}
\label{cmd:dump}
\begin{lstlisting}[numbers=left,frame=single]
    dump [options] [selection]

Write the selected parts of the design to the console or specified file in
ilang format.

    -m
        also dump the module headers, even if only parts of a single
        module is selected

    -n
        only dump the module headers if the entire module is selected

    -outfile <filename>
        Write to the specified file.
\end{lstlisting}

\section{eval -- evaluate the circuit given an input}
\label{cmd:eval}
\begin{lstlisting}[numbers=left,frame=single]
    eval [options] [selection]

This command evaluates the value of a signal given the value of all required
inputs.

    -set <signal> <value>
        set the specified signal to the specified value.

    -set-undef
        set all unspecified source signals to undef (x)

    -table <signal>
        create a truth table using the specified input signals

    -show <signal>
        show the value for the specified signal. if no -show option is passed
        then all output ports of the current module are used.
\end{lstlisting}

\section{extract -- find subcircuits and replace them with cells}
\label{cmd:extract}
\begin{lstlisting}[numbers=left,frame=single]
    extract -map <map_file> [options] [selection]
    extract -mine <out_file> [options] [selection]

This pass looks for subcircuits that are isomorphic to any of the modules
in the given map file and replaces them with instances of this modules. The
map file can be a verilog source file (*.v) or an ilang file (*.il).

    -map <map_file>
        use the modules in this file as reference

    -verbose
        print debug output while analyzing

    -constports
        also find instances with constant drivers. this may be much
        slower than the normal operation.

    -nodefaultswaps
        normally builtin port swapping rules for internal cells are used per
        default. This turns that off, so e.g. 'a^b' does not match 'b^a'
        when this option is used.

    -compat <needle_type> <haystack_type>
        Per default, the cells in the map file (needle) must have the
        type as the cells in the active design (haystack). This option
        can be used to register additional pairs of types that should
        match. This option can be used multiple times.

    -swap <needle_type> <port1>,<port2>[,...]
        Register a set of swapable ports for a needle cell type.
        This option can be used multiple times.

    -perm <needle_type> <port1>,<port2>[,...] <portA>,<portB>[,...]
        Register a valid permutation of swapable ports for a needle
        cell type. This option can be used multiple times.

    -cell_attr <attribute_name>
        Attributes on cells with the given name must match.

    -wire_attr <attribute_name>
        Attributes on wires with the given name must match.

This pass does not operate on modules with uprocessed processes in it.
(I.e. the 'proc' pass should be used first to convert processes to netlists.)

This pass can also be used for mining for frequent subcircuits. In this mode
the following options are to be used instead of the -map option.

    -mine <out_file>
        mine for frequent subcircuits and write them to the given ilang file

    -mine_cells_span <min> <max>
        only mine for subcircuits with the specified number of cells
        default value: 3 5

    -mine_min_freq <num>
        only mine for subcircuits with at least the specified number of matches
        default value: 10

    -mine_limit_matches_per_module <num>
        when calculating the number of matches for a subcircuit, don't count
        more than the specified number of matches per module

    -mine_max_fanout <num>
        don't consider internal signals with more than <num> connections

The modules in the map file may have the attribute 'extract_order' set to an
integer value. Then this value is used to determine the order in which the pass
tries to map the modules to the design (ascending, default value is 0).

See 'help techmap' for a pass that does the opposite thing.
\end{lstlisting}

\section{flatten -- flatten design}
\label{cmd:flatten}
\begin{lstlisting}[numbers=left,frame=single]
    flatten [selection]

This pass flattens the design by replacing cells by their implementation. This
pass is very simmilar to the 'techmap' pass. The only difference is that this
pass is using the current design as mapping library.
\end{lstlisting}

\section{freduce -- perform functional reduction}
\label{cmd:freduce}
\begin{lstlisting}[numbers=left,frame=single]
    freduce [options] [selection]

This pass performs functional reduction in the circuit. I.e. if two nodes are
equivialent, they are merged to one node and one of the redundant drivers is
removed.

    -try
        do not issue an error when the analysis fails.
        (usually beacause of logic loops in the design)
\end{lstlisting}

\section{fsm -- extract and optimize finite state machines}
\label{cmd:fsm}
\begin{lstlisting}[numbers=left,frame=single]
    fsm [options] [selection]

This pass calls all the other fsm_* passes in a useful order. This performs
FSM extraction and optimiziation. It also calls opt_clean as needed:

    fsm_detect          unless got option -nodetect
    fsm_extract

    fsm_opt
    opt_clean
    fsm_opt

    fsm_expand          if got option -expand
    opt_clean           if got option -expand
    fsm_opt             if got option -expand

    fsm_recode          unless got option -norecode

    fsm_info

    fsm_export          if got option -export
    fsm_map             unless got option -nomap

Options:

    -expand, -norecode, -export, -nomap
        enable or disable passes as indicated above

    -encoding tye
    -fm_set_fsm_file file
        passed through to fsm_recode pass
\end{lstlisting}

\section{fsm\_detect -- finding FSMs in design}
\label{cmd:fsm_detect}
\begin{lstlisting}[numbers=left,frame=single]
    fsm_detect [selection]

This pass detects finite state machines by identifying the state signal.
The state signal is then marked by setting the attribute 'fsm_encoding'
on the state signal to "auto".

Existing 'fsm_encoding' attributes are not changed by this pass.

Signals can be protected from being detected by this pass by setting the
'fsm_encoding' attribute to "none".
\end{lstlisting}

\section{fsm\_expand -- expand FSM cells by merging logic into it}
\label{cmd:fsm_expand}
\begin{lstlisting}[numbers=left,frame=single]
    fsm_expand [selection]

The fsm_extract pass is conservative about the cells that belong to a finite
state machine. This pass can be used to merge additional auxiliary gates into
the finate state machine.
\end{lstlisting}

\section{fsm\_export -- exporting FSMs to KISS2 files}
\label{cmd:fsm_export}
\begin{lstlisting}[numbers=left,frame=single]
    fsm_export [-noauto] [-o filename] [-origenc] [selection]

This pass creates a KISS2 file for every selected FSM. For FSMs with the
'fsm_export' attribute set, the attribute value is used as filename, otherwise
the module and cell name is used as filename. If the parameter '-o' is given,
the first exported FSM is written to the specified filename. This overwrites
the setting as specified with the 'fsm_export' attribute. All other FSMs are
exported to the default name as mentioned above.

    -noauto
        only export FSMs that have the 'fsm_export' attribute set

    -o filename
        filename of the first exported FSM

    -origenc
        use binary state encoding as state names instead of s0, s1, ...
\end{lstlisting}

\section{fsm\_extract -- extracting FSMs in design}
\label{cmd:fsm_extract}
\begin{lstlisting}[numbers=left,frame=single]
    fsm_extract [selection]

This pass operates on all signals marked as FSM state signals using the
'fsm_encoding' attribute. It consumes the logic that creates the state signal
and uses the state signal to generate control signal and replaces it with an
FSM cell.

The generated FSM cell still generates the original state signal with its
original encoding. The 'fsm_opt' pass can be used in combination with the
'opt_clean' pass to eliminate this signal.
\end{lstlisting}

\section{fsm\_info -- print information on finite state machines}
\label{cmd:fsm_info}
\begin{lstlisting}[numbers=left,frame=single]
    fsm_info [selection]

This pass dumps all internal information on FSM cells. It can be useful for
analyzing the synthesis process and is called automatically by the 'fsm'
pass so that this information is included in the synthesis log file.
\end{lstlisting}

\section{fsm\_map -- mapping FSMs to basic logic}
\label{cmd:fsm_map}
\begin{lstlisting}[numbers=left,frame=single]
    fsm_map [selection]

This pass translates FSM cells to flip-flops and logic.
\end{lstlisting}

\section{fsm\_opt -- optimize finite state machines}
\label{cmd:fsm_opt}
\begin{lstlisting}[numbers=left,frame=single]
    fsm_opt [selection]

This pass optimizes FSM cells. It detects which output signals are actually
not used and removes them from the FSM. This pass is usually used in
combination with the 'opt_clean' pass (see also 'help fsm').
\end{lstlisting}

\section{fsm\_recode -- recoding finite state machines}
\label{cmd:fsm_recode}
\begin{lstlisting}[numbers=left,frame=single]
    fsm_recode [-encoding type] [-fm_set_fsm_file file] [selection]

This pass reassign the state encodings for FSM cells. At the moment only
one-hot encoding and binary encoding is supported. The option -encoding
can be used to specify the encoding scheme used for FSMs without the
`fsm_encoding' attribute (or with the attribute set to `auto'.

The option -fm_set_fsm_file can be used to generate a file containing the
mapping from old to new FSM encoding in form of Synopsys Formality set_fsm_*
commands.
\end{lstlisting}

\section{help -- display help messages}
\label{cmd:help}
\begin{lstlisting}[numbers=left,frame=single]
    help  .............  list all commands
    help <command>  ...  print help message for given command
    help -all  ........  print complete command reference
\end{lstlisting}

\section{hierarchy -- check, expand and clean up design hierarchy}
\label{cmd:hierarchy}
\begin{lstlisting}[numbers=left,frame=single]
    hierarchy [-check] [-top <module>]
    hierarchy -generate <cell-types> <port-decls>

In parametric designs, a module might exists in serveral variations with
different parameter values. This pass looks at all modules in the current
design an re-runs the language frontends for the parametric modules as
needed.

    -check
        also check the design hierarchy. this generates an error when
        an unknown module is used as cell type.

    -keep_positionals
        per default this pass also converts positional arguments in cells
        to arguments using port names. this option disables this behavior.

    -top <module>
        use the specified top module to built a design hierarchy. modules
        outside this tree (unused modules) are removed.

        when the -top option is used, the 'top' attribute will be set on the
        specified top module. otherwise a module with the 'top' attribute set
        will implicitly be used as top module, if such a module exists.

In -generate mode this pass generates blackbox modules for the given cell
types (wildcards supported). For this the design is searched for cells that
match the given types and then the given port declarations are used to
determine the direction of the ports. The syntax for a port declaration is:

    {i|o|io}[@<num>]:<portname>

Input ports are specified with the 'i' prefix, output ports with the 'o'
prefix and inout ports with the 'io' prefix. The optional <num> specifies
the position of the port in the parameter list (needed when instanciated
using positional arguments). When <num> is not specified, the <portname> can
also contain wildcard characters.

This pass ignores the current selection and always operates on all modules
in the current design.
\end{lstlisting}

\section{history -- show last interactive commands}
\label{cmd:history}
\begin{lstlisting}[numbers=left,frame=single]
    history

This command prints all commands in the shell history buffer. This are
all commands executed in an interactive session, but not the commands
from executed scripts.
\end{lstlisting}

\section{iopadmap -- technology mapping of i/o pads (or buffers)}
\label{cmd:iopadmap}
\begin{lstlisting}[numbers=left,frame=single]
    iopadmap [options] [selection]

Map module inputs/outputs to PAD cells from a library. This pass
can only map to very simple PAD cells. Use 'techmap' to further map
the resulting cells to more sophisticated PAD cells.

    -inpad <celltype> <portname>[:<portname>]
        Map module input ports to the given cell type with
        the given port name. if a 2nd portname is given, the
        signal is passed through the pad call, using the 2nd
        portname as output.

    -outpad <celltype> <portname>[:<portname>]
    -inoutpad <celltype> <portname>[:<portname>]
        Similar to -inpad, but for output and inout ports.

    -widthparam <param_name>
        Use the specified parameter name to set the port width.

    -nameparam <param_name>
        Use the specified parameter to set the port name.
\end{lstlisting}

\section{ls -- list modules or objects in modules}
\label{cmd:ls}
\begin{lstlisting}[numbers=left,frame=single]
    ls [pattern]

When no active module is selected, this prints a list of all modules.

When an active module is selected, this prints a list of objects in the module.

If a pattern is given, the objects matching the pattern are printed

Note that this command does not use the selection mechanism and always operates
on the whole design or whole active module. Use 'select -list' to show a list
of currently selected objects.
\end{lstlisting}

\section{memory -- translate memories to basic cells}
\label{cmd:memory}
\begin{lstlisting}[numbers=left,frame=single]
    memory [-nomap] [selection]

This pass calls all the other memory_* passes in a useful order:

    memory_dff
    memory_collect
    memory_map          (skipped if called with -nomap)

This converts memories to word-wide DFFs and address decoders
or multiport memory blocks if called with the -nomap option.
\end{lstlisting}

\section{memory\_collect -- creating multi-port memory cells}
\label{cmd:memory_collect}
\begin{lstlisting}[numbers=left,frame=single]
    memory_collect [selection]

This pass collects memories and memory ports and creates generic multiport
memory cells.
\end{lstlisting}

\section{memory\_dff -- merge input/output DFFs into memories}
\label{cmd:memory_dff}
\begin{lstlisting}[numbers=left,frame=single]
    memory_dff [selection]

This pass detects DFFs at memory ports and merges them into the memory port.
I.e. it consumes an asynchronous memory port and the flip-flops at its
interface and yields a synchronous memory port.
\end{lstlisting}

\section{memory\_map -- translate multiport memories to basic cells}
\label{cmd:memory_map}
\begin{lstlisting}[numbers=left,frame=single]
    memory_map [selection]

This pass converts multiport memory cells as generated by the memory_collect
pass to word-wide DFFs and address decoders.
\end{lstlisting}

\section{opt -- perform simple optimizations}
\label{cmd:opt}
\begin{lstlisting}[numbers=left,frame=single]
    opt [selection]

This pass calls all the other opt_* passes in a useful order. This performs
a series of trivial optimizations and cleanups. This pass executes the other
passes in the following order:

    opt_const
    opt_share -nomux

    do
        opt_muxtree
        opt_reduce
        opt_share
        opt_rmdff
        opt_clean
        opt_const
    while [changed design]
\end{lstlisting}

\section{opt\_clean -- remove unused cells and wires}
\label{cmd:opt_clean}
\begin{lstlisting}[numbers=left,frame=single]
    opt_clean [options] [selection]

This pass identifies wires and cells that are unused and removes them. Other
passes often remove cells but leave the wires in the design or reconnect the
wires but leave the old cells in the design. This pass can be used to clean up
after the passes that do the actual work.

This pass only operates on completely selected modules without processes.

    -purge
        also remove internal nets if they have a public name
\end{lstlisting}

\section{opt\_const -- perform const folding}
\label{cmd:opt_const}
\begin{lstlisting}[numbers=left,frame=single]
    opt_const [selection]

This pass performs const folding on internal cell types with constant inputs.
\end{lstlisting}

\section{opt\_muxtree -- eliminate dead trees in multiplexer trees}
\label{cmd:opt_muxtree}
\begin{lstlisting}[numbers=left,frame=single]
    opt_muxtree [selection]

This pass analyzes the control signals for the multiplexer trees in the design
and identifies inputs that can never be active. It then removes this dead
branches from the multiplexer trees.

This pass only operates on completely selected modules without processes.
\end{lstlisting}

\section{opt\_reduce -- simplify large MUXes and AND/OR gates}
\label{cmd:opt_reduce}
\begin{lstlisting}[numbers=left,frame=single]
    opt_reduce [selection]

This pass performs two interlinked optimizations:

1. it consolidates trees of large AND gates or OR gates and eliminates
duplicated inputs.

2. it identifies duplicated inputs to MUXes and replaces them with a single
input with the original control signals OR'ed together.
\end{lstlisting}

\section{opt\_rmdff -- remove DFFs with constant inputs}
\label{cmd:opt_rmdff}
\begin{lstlisting}[numbers=left,frame=single]
    opt_rmdff [selection]

This pass identifies flip-flops with constant inputs and replaces them with
a constant driver.
\end{lstlisting}

\section{opt\_share -- consolidate identical cells}
\label{cmd:opt_share}
\begin{lstlisting}[numbers=left,frame=single]
    opt_share [-nomux] [selection]

This pass identifies cells with identical type and input signals. Such cells
are then merged to one cell.

    -nomux
        Do not merge MUX cells.
\end{lstlisting}

\section{proc -- translate processes to netlists}
\label{cmd:proc}
\begin{lstlisting}[numbers=left,frame=single]
    proc [options] [selection]

This pass calls all the other proc_* passes in the most common order.

    proc_clean
    proc_rmdead
    proc_init
    proc_arst
    proc_mux
    proc_dff
    proc_clean

This replaces the processes in the design with multiplexers and flip-flops.

The following options are supported:

    -global_arst [!]<netname>
        This option is passed through to proc_arst.
\end{lstlisting}

\section{proc\_arst -- detect asynchronous resets}
\label{cmd:proc_arst}
\begin{lstlisting}[numbers=left,frame=single]
    proc_arst [-global_arst [!]<netname>] [selection]

This pass identifies asynchronous resets in the processes and converts them
to a different internal representation that is suitable for generating
flip-flop cells with asynchronous resets.

    -global_arst [!]<netname>
        In modules that have a net with the given name, use this net as async
        reset for registers that have been assign initial values in their
        declaration ('reg foobar = constant_value;'). Use the '!' modifier for
        active low reset signals. Note: the frontend stores the default value
        in the 'init' attribute on the net.
\end{lstlisting}

\section{proc\_clean -- remove empty parts of processes}
\label{cmd:proc_clean}
\begin{lstlisting}[numbers=left,frame=single]
    proc_clean [selection]

This pass removes empty parts of processes and ultimately removes a process
if it contains only empty structures.
\end{lstlisting}

\section{proc\_dff -- extract flip-flops from processes}
\label{cmd:proc_dff}
\begin{lstlisting}[numbers=left,frame=single]
    proc_dff [selection]

This pass identifies flip-flops in the processes and converts them to
d-type flip-flop cells.
\end{lstlisting}

\section{proc\_init -- convert initial block to init attributes}
\label{cmd:proc_init}
\begin{lstlisting}[numbers=left,frame=single]
    proc_init [selection]

This pass extracts the 'init' actions from processes (generated from verilog
'initial' blocks) and sets the initial value to the 'init' attribute on the
respective wire.
\end{lstlisting}

\section{proc\_mux -- convert decision trees to multiplexers}
\label{cmd:proc_mux}
\begin{lstlisting}[numbers=left,frame=single]
    proc_mux [selection]

This pass converts the decision trees in processes (originating from if-else
and case statements) to trees of multiplexer cells.
\end{lstlisting}

\section{proc\_rmdead -- eliminate dead trees in decision trees}
\label{cmd:proc_rmdead}
\begin{lstlisting}[numbers=left,frame=single]
    proc_rmdead [selection]

This pass identifies unreachable branches in decision trees and removes them.
\end{lstlisting}

\section{read\_ilang -- read modules from ilang file}
\label{cmd:read_ilang}
\begin{lstlisting}[numbers=left,frame=single]
    read_ilang [filename]

Load modules from an ilang file to the current design. (ilang is a text
representation of a design in yosys's internal format.)
\end{lstlisting}

\section{read\_verilog -- read modules from verilog file}
\label{cmd:read_verilog}
\begin{lstlisting}[numbers=left,frame=single]
    read_verilog [filename]

Load modules from a verilog file to the current design. A large subset of
Verilog-2005 is supported.

    -dump_ast1
        dump abstract syntax tree (before simplification)

    -dump_ast2
        dump abstract syntax tree (after simplification)

    -dump_vlog
        dump ast as verilog code (after simplification)

    -yydebug
        enable parser debug output

    -nolatches
        usually latches are synthesized into logic loops
        this option prohibits this and sets the output to 'x'
        in what would be the latches hold condition

        this behavior can also be achieved by setting the
        'nolatches' attribute on the respective module or
        always block.

    -nomem2reg
        under certain conditions memories are converted to registers
        early during simplification to ensure correct handling of
        complex corner cases. this option disables this behavior.

        this can also be achieved by setting the 'nomem2reg'
        attribute on the respective module or register.

    -mem2reg
        always convert memories to registers. this can also be
        achieved by setting the 'mem2reg' attribute on the respective
        module or register.

    -ppdump
        dump verilog code after pre-processor

    -nopp
        do not run the pre-processor

    -lib
        only create empty blackbox modules

    -noopt
        don't perform basic optimizations (such as const folding) in the
        high-level front-end.

    -ignore_redef
        ignore re-definitions of modules. (the default behavior is to
        create an error message.)

    -Dname[=definition]
        define the preprocessor symbol 'name' and set its optional value
        'definition'

    -Idir
        add 'dir' to the directories which are used when searching include
        files
\end{lstlisting}

\section{rename -- rename object in the design}
\label{cmd:rename}
\begin{lstlisting}[numbers=left,frame=single]
    rename old_name new_name

Rename the specified object. Note that selection patterns are not supported
by this command.


    rename -enumerate [selection]

Assign short auto-generated names to all selected wires and cells with private
names.
\end{lstlisting}

\section{sat -- solve a SAT problem in the circuit}
\label{cmd:sat}
\begin{lstlisting}[numbers=left,frame=single]
    sat [options] [selection]

This command solves a SAT problem defined over the currently selected circuit
and additional constraints passed as parameters.

    -all
        show all solutions to the problem (this can grow exponentially, use
        -max <N> instead to get <N> solutions)

    -max <N>
        like -all, but limit number of solutions to <N>

    -enable_undef
        enable modeling of undef value (aka 'x-bits')
        this option is implied by -set-def, -set-undef et. cetera

    -max_undef
        maximize the number of undef bits in solutions, giving a better
        picture of which input bits are actually vital to the solution.

    -set <signal> <value>
        set the specified signal to the specified value.

    -set-def <signal>
        add a constraint that all bits of the given signal must be defined

    -set-any-undef <signal>
        add a constraint that at least one bit of the given signal is undefined

    -set-all-undef <signal>
        add a constraint that all bits of the given signal are undefined

    -set-def-inputs
        add -set-def constraints for all module inputs

    -show <signal>
        show the model for the specified signal. if no -show option is
        passed then a set of signals to be shown is automatically selected.

    -ignore_div_by_zero
        ignore all solutions that involve a division by zero

The following options can be used to set up a sequential problem:

    -seq <N>
        set up a sequential problem with <N> time steps. The steps will
        be numbered from 1 to N.

    -set-at <N> <signal> <value>
    -unset-at <N> <signal>
        set or unset the specified signal to the specified value in the
        given timestep. this has priority over a -set for the same signal.

    -set-def-at <N> <signal>
    -set-any-undef-at <N> <signal>
    -set-all-undef-at <N> <signal>
        add undef contraints in the given timestep.

    -set-init <signal> <value>
        set the initial value for the register driving the signal to the value

    -set-init-undef
        set all initial states (not set using -set-init) to undef

The following additional options can be used to set up a proof. If also -seq
is passed, a temporal induction proof is performed.

    -prove <signal> <value>
        Attempt to proof that <signal> is always <value>. In a temporal
        induction proof it is proven that the condition holds forever after
        the number of time steps passed using -seq.

    -prove-x <signal> <value>
        Like -prove, but an undef (x) bit in the lhs matches any value on
        the right hand side. Useful for equivialence checking.

    -maxsteps <N>
        Set a maximum length for the induction.

    -timeout <N>
        Maximum number of seconds a single SAT instance may take.

    -verify
        Return an error and stop the synthesis script if the proof fails.

    -verify-no-timeout
        Like -verify but do not return an error for timeouts.
\end{lstlisting}

\section{scatter -- add additional intermediate nets}
\label{cmd:scatter}
\begin{lstlisting}[numbers=left,frame=single]
    scatter [selection]

This command adds additional intermediate nets on all cell ports. This is used
for testing the correct use of the SigMap helper in passes. If you don't know
what this means: don't worry -- you only need this pass when testing your own
extensions to Yosys.

Use the opt_clean command to get rid of the additional nets.
\end{lstlisting}

\section{scc -- detect strongly connected components (logic loops)}
\label{cmd:scc}
\begin{lstlisting}[numbers=left,frame=single]
    scc [options] [selection]

This command identifies strongly connected components (aka logic loops) in the
design.

    -max_depth <num>
        limit to loops not longer than the specified number of cells. This can
        e.g. be useful in identifying local loops in a module that turns out
        to be one gigantic SCC.

    -all_cell_types
        Usually this command only considers internal non-memory cells. With
        this option set, all cells are considered. For unkown cells all ports
        are assumed to be bidirectional 'inout' ports.

    -set_attr <name> <value>
    -set_cell_attr <name> <value>
    -set_wire_attr <name> <value>
        set the specified attribute on all cells and/or wires that are part of
        a logic loop. the special token {} in the value is replaced with a
        unique identifier for the logic loop.

    -select
        replace the current selection with a selection of all cells and wires
        that are part of a found logic loop
\end{lstlisting}

\section{script -- execute commands from script file}
\label{cmd:script}
\begin{lstlisting}[numbers=left,frame=single]
    script <filename>

This command executes the yosys commands in the specified file.
\end{lstlisting}

\section{select -- modify and view the list of selected objects}
\label{cmd:select}
\begin{lstlisting}[numbers=left,frame=single]
    select [ -add | -del | -set <name> ] <selection>
    select [ -list | -write <filename> | -count | -clear ]
    select -module <modname>

Most commands use the list of currently selected objects to determine which part
of the design to operate on. This command can be used to modify and view this
list of selected objects.

Note that many commands support an optional [selection] argument that can be
used to override the global selection for the command. The syntax of this
optional argument is identical to the syntax of the <selection> argument
described here.

    -add, -del
        add or remove the given objects to the current selection.
        without this options the current selection is replaced.

    -set <name>
        do not modify the current selection. instead save the new selection
        under the given name (see @<name> below).

    -list
        list all objects in the current selection

    -write <filename>
        like -list but write the output to the specified file

    -count
        count all objects in the current selection

    -clear
        clear the current selection. this effectively selects the
        whole design.

    -module <modname>
        limit the current scope to the specified module.
        the difference between this and simply selecting the module
        is that all object names are interpreted relative to this
        module after this command until the selection is cleared again.

When this command is called without an argument, the current selection
is displayed in a compact form (i.e. only the module name when a whole module
is selected).

The <selection> argument itself is a series of commands for a simple stack
machine. Each element on the stack represents a set of selected objects.
After this commands have been executed, the union of all remaining sets
on the stack is computed and used as selection for the command.

Pushing (selecting) object when not in -module mode:

    <mod_pattern>
        select the specified module(s)

    <mod_pattern>/<obj_pattern>
        select the specified object(s) from the module(s)

Pushing (selecting) object when in -module mode:

    <obj_pattern>
        select the specified object(s) from the current module

A <mod_pattern> can be a module name or wildcard expression (*, ?, [..])
matching module names.

An <obj_pattern> can be an object name, wildcard expression, or one of
the following:

    w:<pattern>
        all wires with a name matching the given wildcard pattern

    m:<pattern>
        all memories with a name matching the given pattern

    c:<pattern>
        all cells with a name matching the given pattern

    t:<pattern>
        all cells with a type matching the given pattern

    p:<pattern>
        all processes with a name matching the given pattern

    a:<pattern>
        all objects with an attribute name matching the given pattern

    a:<pattern>=<pattern>
        all objects with a matching attribute name-value-pair

    n:<pattern>
        all objects with a name matching the given pattern
        (i.e. 'n:' is optional as it is the default matching rule)

    @<name>
        push the selection saved prior with 'select -set <name> ...'

The following actions can be performed on the top sets on the stack:

    %
        push a copy of the current selection to the stack

    %%
        replace the stack with a union of all elements on it

    %n
        replace top set with its invert

    %u
        replace the two top sets on the stack with their union

    %i
        replace the two top sets on the stack with their intersection

    %d
        pop the top set from the stack and subtract it from the new top

    %x[<num1>|*][.<num2>][:<rule>[:<rule>..]]
        expand top set <num1> num times according to the specified rules.
        (i.e. select all cells connected to selected wires and select all
        wires connected to selected cells) The rules specify which cell
        ports to use for this. the syntax for a rule is a '-' for exclusion
        and a '+' for inclusion, followed by an optional comma seperated
        list of cell types followed by an optional comma separated list of
        cell ports in square brackets. a rule can also be just a cell or wire
        name that limits the expansion (is included but does not go beyond).
        select at most <num2> objects. a warning message is printed when this
        limit is reached. When '*' is used instead of <num1> then the process
        is repeated until no further object are selected.

    %ci[<num1>|*][.<num2>][:<rule>[:<rule>..]]
    %co[<num1>|*][.<num2>][:<rule>[:<rule>..]]
        simmilar to %x, but only select input (%ci) or output cones (%co)

Example: the following command selects all wires that are connected to a
'GATE' input of a 'SWITCH' cell:

    select */t:SWITCH %x:+[GATE] */t:SWITCH %d
\end{lstlisting}

\section{shell -- enter interactive command mode}
\label{cmd:shell}
\begin{lstlisting}[numbers=left,frame=single]
    shell

This command enters the interactive command mode. This can be useful
in a script to interrupt the script at a certain point and allow for
interactive inspection or manual synthesis of the design at this point.

The command prompt of the interactive shell indicates the current
selection (see 'help select'):

    yosys>
        the entire design is selected

    yosys*>
        only part of the design is selected

    yosys [modname]>
        the entire module 'modname' is selected using 'select -module modname'

    yosys [modname]*>
        only part of current module 'modname' is selected

When in interactive shell, some errors (e.g. invalid command arguments)
do not terminate yosys but return to the command prompt.

This command is the default action if nothing else has been specified
on the command line.

Press Ctrl-D or type 'exit' to leave the interactive shell.
\end{lstlisting}

\section{show -- generate schematics using graphviz}
\label{cmd:show}
\begin{lstlisting}[numbers=left,frame=single]
    show [options] [selection]

Create a graphviz DOT file for the selected part of the design and compile it
to a graphics file (usually SVG or PostScript).

    -viewer <viewer>
        Run the specified command with the graphics file as parameter.

    -format <format>
        Generate a graphics file in the specified format.
        Usually <format> is 'svg' or 'ps'.

    -lib <verilog_or_ilang_file>
        Use the specified library file for determining whether cell ports are
        inputs or outputs. This option can be used multiple times to specify
        more than one library.

    -prefix <prefix>
        generate <prefix>.* instead of ~/.yosys_show.*

    -color <color> <wire>
        assign the specified color to the specified wire. The object can be
        a single selection wildcard expressions or a saved set of objects in
        the @<name> syntax (see "help select" for details).

    -colors <seed>
        Randomly assign colors to the wires. The integer argument is the seed
        for the random number generator. Change the seed value if the colored
        graph still is ambigous. A seed of zero deactivates the coloring.

    -width
        annotate busses with a label indicating the width of the bus.

    -stretch
        stretch the graph so all inputs are on the left side and all outputs
        (including inout ports) are on the right side.

    -pause
        wait for the use to press enter to before returning

    -enum
        enumerate objects with internal ($-prefixed) names

    -long
        do not abbeviate objects with internal ($-prefixed) names

When no <format> is specified, SVG is used. When no <format> and <viewer> is
specified, 'yosys-svgviewer' is used to display the schematic.

The generated output files are '~/.yosys_show.dot' and '~/.yosys_show.<format>',
unless another prefix is specified using -prefix <prefix>.
\end{lstlisting}

\section{simplemap -- mapping simple coarse-grain cells}
\label{cmd:simplemap}
\begin{lstlisting}[numbers=left,frame=single]
    simplemap [selection]

This pass maps a small selection of simple coarse-grain cells to yosys gate
primitives. The following internal cell types are mapped by this pass:

  $not, $pos, $bu0, $and, $or, $xor, $xnor
  $reduce_and, $reduce_or, $reduce_xor, $reduce_xnor, $reduce_bool
  $logic_not, $logic_and, $logic_or, $mux
  $sr, $dff, $dffsr, $adff, $dlatch
\end{lstlisting}

\section{splitnets -- split up multi-bit nets}
\label{cmd:splitnets}
\begin{lstlisting}[numbers=left,frame=single]
    splitnets [options] [selection]

This command splits multi-bit nets into single-bit nets.

    -format char1[char2]
        the first char is inserted between the net name and the bit index, the
        second char is appended to the netname. e.g. -format () creates net
        names like 'mysignal(42)'. the default is '[]'.

    -ports
        also split module ports. per default only internal signals are split.
\end{lstlisting}

\section{stat -- print some statistics}
\label{cmd:stat}
\begin{lstlisting}[numbers=left,frame=single]
    stat [options] [selection]

Print some statistics (number of objects) on the selected portion of the
design.

    -top <module>
        print design hierarchy with this module as top. if the design is fully
        selected and a module has the 'top' attribute set, this module is used
        default value for this option.
\end{lstlisting}

\section{submod -- moving part of a module to a new submodule}
\label{cmd:submod}
\begin{lstlisting}[numbers=left,frame=single]
    submod [selection]

This pass identifies all cells with the 'submod' attribute and moves them to
a newly created module. The value of the attribute is used as name for the
cell that replaces the group of cells with the same attribute value.

This pass can be used to create a design hierarchy in flat design. This can
be useful for analyzing or reverse-engineering a design.

This pass only operates on completely selected modules with no processes
or memories.


    submod -name <name> [selection]

As above, but don't use the 'submod' attribute but instead use the selection.
Only objects from one module might be selected. The value of the -name option
is used as the value of the 'submod' attribute above.
\end{lstlisting}

\section{synth\_xilinx -- synthesis for Xilinx FPGAs}
\label{cmd:synth_xilinx}
\begin{lstlisting}[numbers=left,frame=single]
    synth_xilinx [options]

This command runs synthesis for Xilinx FPGAs. This command does not operate on
partly selected designs.

    -top <module>
        use the specified module as top module (default='top')

    -arch <arch>
        select architecture. the following architectures are supported:
            spartan6 (default), artix7, kintex7, virtex7, zynq7000
            (this parameter is not used by the command at the moment)

    -edif <file>
        write the design to the specified edif file. writing of an output file
        is omitted if this parameter is not specified.

    -run <from_label>:<to_label>
        only run the commands between the labels (see below). an empty
        from label is synonymous to 'begin', and empty to label is
        synonymous to the end of the command list.


The following commands are executed by this synthesis command:

    begin:
        hierarchy -check -top <top>

    coarse:
        proc
        opt
        memory
        clean
        fsm
        opt

    fine:
        techmap
        opt

    map_luts:
        abc -lut 6
        clean

    map_cells:
        techmap -share_map xilinx/cells.v
        clean

    clkbuf:
        select -set xilinx_clocks <top>/t:FDRE %x:+FDRE[C] <top>/t:FDRE %d
        iopadmap -inpad BUFGP O:I @xilinx_clocks

    iobuf:
        select -set xilinx_nonclocks <top>/w:* <top>/t:BUFGP %x:+BUFGP[I] %d
        iopadmap -outpad OBUF I:O -inpad IBUF O:I @xilinx_nonclocks

    edif:
        write_edif synth.edif
\end{lstlisting}

\section{tcl -- execute a TCL script file}
\label{cmd:tcl}
\begin{lstlisting}[numbers=left,frame=single]
    tcl <filename>

This command executes the tcl commands in the specified file.
Use 'yosys cmd' to run the yosys command 'cmd' from tcl.

The tcl command 'yosys -import' can be used to import all yosys
commands directly as tcl commands to the tcl shell. The yosys
command 'proc' is wrapped using the tcl command 'procs' in order
to avoid a name collision with the tcl builting command 'proc'.
\end{lstlisting}

\section{techmap -- simple technology mapper}
\label{cmd:techmap}
\begin{lstlisting}[numbers=left,frame=single]
    techmap [-map filename] [selection]

This pass implements a very simple technology mapper that replaces cells in
the design with implementations given in form of a verilog or ilang source
file.

    -map filename
        the library of cell implementations to be used.
        without this parameter a builtin library is used that
        transforms the internal RTL cells to the internal gate
        library.

    -share_map filename
        like -map, but look for the file in the share directory (where the
        yosys data files are). this is mainly used internally when techmap
        is called from other commands.

    -D <define>, -I <incdir>
        this options are passed as-is to the verilog frontend for loading the
        map file. Note that the verilog frontend is also called with the
        '-ignore_redef' option set.

When a module in the map file has the 'techmap_celltype' attribute set, it will
match cells with a type that match the text value of this attribute. Otherwise
the module name will be used to match the cell.

When a module in the map file has the 'techmap_simplemap' attribute set, techmap
will use 'simplemap' (see 'help simplemap') to map cells matching the module.

All wires in the modules from the map file matching the pattern _TECHMAP_*
or *._TECHMAP_* are special wires that are used to pass instructions from
the mapping module to the techmap command. At the moment the following special
wires are supported:

    _TECHMAP_FAIL_
        When this wire is set to a non-zero constant value, techmap will not
        use this module and instead try the next module with a matching
        'techmap_celltype' attribute.

        When such a wire exists but does not have a constant value after all
        _TECHMAP_DO_* commands have been executed, an error is generated.

    _TECHMAP_DO_*
        This wires are evaluated in alphabetical order. The constant text value
        of this wire is a yosys command (or sequence of commands) that is run
        by techmap on the module. A common use case is to run 'proc' on modules
        that are written using always-statements.

        When such a wire has a non-constant value at the time it is to be
        evaluated, an error is produced. That means it is possible for such a
        wire to start out as non-constant and evaluate to a constant value
        during processing of other _TECHMAP_DO_* commands.

In addition to this special wires, techmap also supports special parameters in
modules in the map file:

    _TECHMAP_CELLTYPE_
        When a parameter with this name exists, it will be set to the type name
        of the cell that matches the module.

When a module in the map file has a parameter where the according cell in the
design has a port, the module from the map file is only used if the port in
the design is connected to a constant value. The parameter is then set to the
constant value.

See 'help extract' for a pass that does the opposite thing.

See 'help flatten' for a pass that does flatten the design (which is
esentially techmap but using the design itself as map library).
\end{lstlisting}

\section{write\_autotest -- generate simple test benches}
\label{cmd:write_autotest}
\begin{lstlisting}[numbers=left,frame=single]
    write_autotest [filename]

Automatically create primitive verilog test benches for all modules in the
design. The generated testbenches toggle the input pins of the module in
a semi-random manner and dumps the resulting output signals.

This can be used to check the synthesis results for simple circuits by
comparing the testbench output for the input files and the synthesis results.

The backend automatically detects clock signals. Additionally a signal can
be forced to be interpreted as clock signal by setting the attribute
'gentb_clock' on the signal.

The attribute 'gentb_constant' can be used to force a signal to a constant
value after initialization. This can e.g. be used to force a reset signal
low in order to explore more inner states in a state machine.
\end{lstlisting}

\section{write\_blif -- write design to BLIF file}
\label{cmd:write_blif}
\begin{lstlisting}[numbers=left,frame=single]
    write_blif [options] [filename]

Write the current design to an BLIF file.

    -top top_module
        set the specified module as design top module

    -buf <cell-type> <in-port> <out-port>
        use cells of type <cell-type> with the specified port names for buffers

    -true <cell-type> <out-port>
    -false <cell-type> <out-port>
        use the specified cell types to drive nets that are constant 1 or 0

The following options can be usefull when the generated file is not going to be
read by a BLIF parser but a custom tool. It is recommended to not name the output
file *.blif when any of this options is used.

    -subckt
        do not translate Yosys's internal gates to generic BLIF logic
        functions. Instead create .subckt lines for all cells.

    -conn
        do not generate buffers for connected wires. instead use the
        non-standard .conn statement.

    -impltf
        do not write definitions for the $true and $false wires.
\end{lstlisting}

\section{write\_edif -- write design to EDIF netlist file}
\label{cmd:write_edif}
\begin{lstlisting}[numbers=left,frame=single]
    write_edif [options] [filename]

Write the current design to an EDIF netlist file.

    -top top_module
        set the specified module as design top module

Unfortunately there are different "flavors" of the EDIF file format. This
command generates EDIF files for the Xilinx place&route tools. It might be
necessary to make small modifications to this command when a different tool
is targeted.
\end{lstlisting}

\section{write\_ilang -- write design to ilang file}
\label{cmd:write_ilang}
\begin{lstlisting}[numbers=left,frame=single]
    write_ilang [filename]

Write the current design to an 'ilang' file. (ilang is a text representation
of a design in yosys's internal format.)

    -selected
        only write selected parts of the design.
\end{lstlisting}

\section{write\_intersynth -- write design to InterSynth netlist file}
\label{cmd:write_intersynth}
\begin{lstlisting}[numbers=left,frame=single]
    write_intersynth [options] [filename]

Write the current design to an 'intersynth' netlist file. InterSynth is
a tool for Coarse-Grain Example-Driven Interconnect Synthesis.

    -notypes
        do not generate celltypes and conntypes commands. i.e. just output
        the netlists. this is used for postsilicon synthesis.

    -lib <verilog_or_ilang_file>
        Use the specified library file for determining whether cell ports are
        inputs or outputs. This option can be used multiple times to specify
        more than one library.

    -selected
        only write selected modules. modules must be selected entirely or
        not at all.

http://www.clifford.at/intersynth/
\end{lstlisting}

\section{write\_spice -- write design to SPICE netlist file}
\label{cmd:write_spice}
\begin{lstlisting}[numbers=left,frame=single]
    write_spice [options] [filename]

Write the current design to an SPICE netlist file.

    -big_endian
        generate multi-bit ports in MSB first order 
        (default is LSB first)

    -neg net_name
        set the net name for constant 0 (default: Vss)

    -pos net_name
        set the net name for constant 1 (default: Vdd)

    -nc_prefix
        prefix for not-connected nets (default: _NC)

    -top top_module
        set the specified module as design top module
\end{lstlisting}

\section{write\_verilog -- write design to verilog file}
\label{cmd:write_verilog}
\begin{lstlisting}[numbers=left,frame=single]
    write_verilog [options] [filename]

Write the current design to a verilog file.

    -norename
        without this option all internal object names (the ones with a dollar
        instead of a backslash prefix) are changed to short names in the
        format '_<number>_'.

    -noattr
        with this option no attributes are included in the output

    -attr2comment
        with this option attributes are included as comments in the output

    -noexpr
        without this option all internal cells are converted to verilog
        expressions.

    -blackboxes
        usually modules with the 'blackbox' attribute are ignored. with
        this option set only the modules with the 'blackbox' attribute
        are written to the output file.

    -selected
        only write selected modules. modules must be selected entirely or
        not at all.
\end{lstlisting}

